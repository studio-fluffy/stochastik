\documentclass{beamer}
\usetheme{Warsaw}

\usepackage[utf8]{inputenc}
\usepackage{fancybox}
\usepackage{multimedia} 
\usepackage{subfig}
\usepackage{amsmath}
\usepackage{hyperref}
\hypersetup{
    colorlinks=true,     
    urlcolor=blue
}
\usepackage[all]{xy}
\begin{document}


\title[Stochastik] % (optional, only for long titles)
{Stochastik für Informatiker
\\
\includegraphics[scale=0.5]{img/craps}
}
\subtitle{}
\author[Dr. Johannes Riesterer] % (optional, for multiple authors)
{Dr.  rer. nat. Johannes Riesterer}

\date[KPT 2004] % (optional)
{}

\subject{Stochastik}

\frame{\titlepage}

\begin{frame}
    \frametitle{Zentraler Grenzwertsatz}
\framesubtitle{}

\begin{block}{Motivation}
Welche Verteilung hat das arithmetische Mittel $S_n:= \frac{1}{n} \sum_{i=1}^n X_i$?
\end{block}
\begin{block}{Motivation}
Wie und gegen was konvergiert $P_{S_n}$? 
\end{block}

 \end{frame}




\begin{frame}
    \frametitle{Zentraler Grenzwertsatz}
\framesubtitle{}

\begin{block}{Konvergenz von W-Maßen}
Was bedeutet Konvergenz einer Folge von Wahrscheinlichkeitsmaßen?
\end{block}
\begin{block}{Inspiration: Gleichmässige Konvergenz von Funktionen}
Eine Folge von Funktionen $f_n: A \subset \mathbb{R}^n \to \mathbb{R}$ konvergiert gleichmässig gegen eine Funktion $f$, falls 
\begin{align*}
\lim_{n \to \infty} ||f_n(x) -f(x) || = 0
\end{align*}
für alle $x \in A$.
\end{block}

 \end{frame}

\begin{frame}
    \frametitle{Allgemeine Wahrscheinlichketisräume/Nachtrag}
\framesubtitle{}

\begin{block}{Konvergenz von W-Maßen}
Sei $(\Omega, \mathcal{A})$ ein Wahrscheinlichkeitsraum und $P_n : \Omega \to [0,1]$ eine folge von Wahrscheinlichkeits-Maßen. Die Folge konvergiert gegen
das Wahrscheinlichkeits-Maß $P: \Omega \to [0,1]$, falls 
\begin{align*}
\lim_{n \to \infty} \int_\Omega f dP_n = \int_\Omega f dP
\end{align*}
für alle messbaren Funktionen $f: \Omega \to \mathbb{R}$.
\end{block}
 \end{frame}

\begin{frame}
    \frametitle{Highlight}
\framesubtitle{}
\begin{figure}[htp]
      \centering
    \includegraphics[width=0.9\textwidth]{img/firework}
\end{figure}
 \end{frame}


\begin{frame}
    \frametitle{Zentraler Grenzwertsatz}
\framesubtitle{}


\begin{block}{Zentraler Grenzwertsatz}
Sei $(\Omega, \mathcal{A}, P)$ ein Wahrscheinlichkeitsraum und $X_n :  \Omega \to \mathbb{R}$  eine folge stochastisch unabhängiger, identisch verteilter, reeller Zufallsvariablen mit $E(X_n) = \mu$ und $V(X_n)= \sigma^2$. Dann gilt für das arithmetische Mittel $S_n:= \frac{1}{n} \sum_{i=1}^n X_i$
\begin{align*}
P_{ \frac{\sqrt{n}}{\sigma} (S_n-\mu)} \to P_{N(0,1)}
\end{align*}
wobei $ P_{N(0,1)}$ das Wahrscheinlichkeits-Maß mit der Dichte $ \frac {1}{ \sqrt{2\pi}}e^{- \frac {1}{2} x^2}$ ist.
\end{block}

 \end{frame}



\begin{frame}
    \frametitle{Zentraler Grenzwertsatz}
\framesubtitle{}

\begin{block}{Erzeugende Funktion}
Sei $(\Omega, \mathcal{A}, P)$ ein Wahrscheinlichkeitsraum und $X :  \Omega \to \mathbb{R}$  eine reelle Zufallsvariable. Dann heißt die Funktion 
\begin{align*}
\psi_X(t) := \mathbb{E}(e^{tX}), \; t \in I \subset \mathbb{R}
\end{align*}
erzeugende Funktion zu $X$ bzw. $P_X$.
\end{block}

\begin{block}{Stetigkeitssatz von  Lévy }
Sei $(\Omega, \mathcal{A}, P)$ ein Wahrscheinlichkeitsraum so wie  $X$ und $X_n :  \Omega \to \mathbb{R}$   reelle Zufallsvariablen mit erzeugenden Funktionen $\psi$ und $\psi_{n}$. Dann gilt:
\begin{align*}
\psi_n \to \psi \Rightarrow P_{X_n} \to P_X 
\end{align*}
\end{block}

 \end{frame}

\begin{frame}
    \frametitle{Zentraler Grenzwertsatz}
\framesubtitle{}
\begin{block}{Eigenschaften erzeugender Funktionen}
\begin{itemize}
\item $\psi_X(t) = \sum_{k= 0}^n \frac{\mathbb{E}(X^k)}{k!} t^k$ für $|t| \leq \delta$ (Taylor).
\item $e^{\frac{t^2}{2}}$ ist die erzeugende Funktion von $ P_{N(0,1)}$.
\item $\psi_{X +Y} = \psi_X \cdot \psi_Y$
\end{itemize}

\end{block}


 \end{frame}




\begin{frame}
    \frametitle{Zentraler Grenzwertsatz}
\framesubtitle{}

\begin{block}{Beweis Zentraler Grenzwertsatz}
\begin{itemize}
\item $|t| \leq \delta$
\item $\psi(t)$ erzeugende Funktion von $X_n$.
\item $Y_n := \frac{X_n - \mu}{\sigma}$. Dann ist $\mathbb{E}(Y_n) = 0$ und $\mathbb{V}(Y_n) = 1$.
\item $\psi^*(t)$ erzeugende Funktion von $Y_n$.
\item $\psi_n(t)$ erzeugende Funktion von $\frac{Y_n}{\sqrt{n}}$. Dann ist $\psi_n(t) = \psi^*(\frac{t}{\sqrt{n}})$ 
\end{itemize}
\begin{align*}
\psi_n(t) &= \psi^*(\frac{t}{\sqrt{n}}) =  \sum_{k= 0}^n \frac{t^k }{k! \sqrt{n}^k} \mathbb{E}(Y_i^k) \\
& = 1 + \frac{t^2}{2n} +  \underbrace{\sum_{k= 3}^n \frac{t^k }{k! \sqrt{n}^k} \mathbb{E}(Y_i^k)}_{=:R_n} 
\end{align*}
\end{block}

 \end{frame}



\begin{frame}
    \frametitle{Zentraler Grenzwertsatz}
\framesubtitle{}

\begin{block}{Beweis Zentraler Grenzwertsatz}
\begin{align*}
R_n \leq \frac{1}{n \sqrt{n}} (\psi^*(\delta)  + \psi^*(-\delta) ) \rightarrow 0  \text{ für } n \to \infty
\end{align*}
Für $T_n := \frac{\sqrt{n}}{\sigma}(S_n - \mu)$ erhält man damit
\begin{align*}
\psi_{T_n}(t) = (\psi_n)(t))^n = \biggl(  1 + \frac{t^2}{2n} + R_n(t) \biggr)^n \rightarrow e^{\frac{t^2}{2}} \text{ für } n \to \infty
\end{align*}
Mit dem Stetigkeitssatz von Levy folgt der zentrale Grentzwertsatz.
\end{block}

 \end{frame}



\begin{frame}
    \frametitle{Zentraler Grenzwertsatz}
\framesubtitle{}

\begin{block}{Umkehrsatz}
Ist $f: \mathbb{R}^n  \to  \mathbb{R}$ und $\hat{ f}$ integrierbar, gilt
\begin{align*}
f(x) = \frac{1}{\left(2\pi \right)^{n/2}} \int_{\mathbb{R}^n}\hat{ f}(y) \,e^{i  <x, y>} \, d y,
\end{align*}
fast überall.
\end{block}
 \end{frame}



\begin{frame}
    \frametitle{Zentraler Grenzwertsatz}
\framesubtitle{}

\begin{block}{Stetigkeitssatz von Levy}
Mit $\varphi_{X} := \mathbb{E}(e^{i tx})$  ist $\varphi _{X}(-it)=\psi_{X}(t)$ und der Stetigkeitssatz von Levy folgt aus dem Umkehrsatz.
\end{block}
 \end{frame}



\begin{frame}
    \frametitle{Highlight}
\framesubtitle{}
\begin{figure}[htp]
      \centering
    \includegraphics[width=0.9\textwidth]{img/firework}
\end{figure}
 \end{frame}


\begin{frame}
    \frametitle{Zentraler Grenzwertsatz}
\framesubtitle{}


\begin{block}{Sensorrauschen}
Wir können annehmen, dass das Rauschen eines Sensor aus vielen kleinen, stochastisch Unabhängigen Effekten Beruht, die sich aufsummieren.  Diese Summe ist nach dem zentralen Grenzwertsatz näherungsweise Normalverteilt.
\end{block}

 \end{frame}


\end{document}
