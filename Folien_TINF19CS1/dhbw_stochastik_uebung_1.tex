\documentclass{beamer}
\usetheme{Warsaw}

\usepackage[utf8]{inputenc}
\usepackage{fancybox}
\usepackage{multimedia} 
\usepackage{subfig}
\usepackage{amsmath}
\usepackage{hyperref}
\usepackage[all]{xy}
\begin{document}


\title[Stochastik] % (optional, only for long titles)
{Stochastik für Informatiker
\\
\includegraphics[scale=0.5]{img/craps}
}
\subtitle{}
\author[Dr. Johannes Riesterer] % (optional, for multiple authors)
{Dr.  rer. nat. Johannes Riesterer}

\date[KPT 2004] % (optional)
{}

\subject{Stochastik}

\frame{\titlepage}

\begin{frame}
    \frametitle{Übungsaufgaben}
\framesubtitle{}
\begin{block}{Aufgabe}
Beim Lottospiel werden ohne Zurücklegen $6$ Zahlen aus $49$ gezogen. Berechnen Sie die folgenden Wahrscheinlichkeiten:
\begin{enumerate}
\item Alle $6$ Gewinnzahlen richtig zu tippen.
\item Genau $5$ richtige Gewinnzahlen zu tippen.
\item Mindestens $3$ richtige Gewinnzahlen zu tippen.
\item Alle $6$ Gewinnzahlen sind gerade.
\end{enumerate}
\end{block}
 \end{frame}



\begin{frame}
    \frametitle{Übungsaufgaben}
\framesubtitle{}
\begin{block}{Aufgabe}
Drei Bits werden über einen digitalen Nachrichtenkanal übertragen. Jedes Bit kann falsch oder richtig empfangen werden.
\begin{enumerate}
\item Geben Sie den Ereignisraum (Grundmenge) $\Omega$ an.
\item Wie viele Elemente besitzt $\Omega$?
\item Es sei $A_i := \{ \text{ i-tes Bit ist verfälscht}\}$. Geben Sie das Ereignis $A_1$ an.
\end{enumerate}
\end{block}

 \end{frame}




\begin{frame}
    \frametitle{Übungsaufgaben}
\framesubtitle{}
\begin{block}{Aufgabe}
Es sei $\Omega = \{ 1,2,3,4\}$. 
\begin{enumerate}
\item Welche der folgenden Mengen sind $\sigma$-Algebren?
\begin{align*}
& A =  \{ \emptyset,  \Omega  \} \\
& B=  \{ \emptyset,  \Omega , \{ 1\}, \{ 2,3\}, \{ 4\} \}  \\
& C=  \{ \emptyset,  \Omega ,  \{ 1,2\}, \{ 3, 4\} \}  
\end{align*}
\item Geben Sie die kleinste Sigma-Algebra über $\Omega$ an, in der die Mengen $ \{ 1\}$ und $ \{ 2\}$ enthalten sind.
\end{enumerate}
\end{block}

 \end{frame}




\begin{frame}
    \frametitle{Übungsaufgaben}
\framesubtitle{}
\begin{block}{Aufgabe}
Bei einer Qualitätskontrolle können Werkstücke zwei Fehler haben, den Fehler $A$ und den Fehler $B$. Aus Erfahrung ist bekannt, dass ein Werkstück mit Wahrscheinlichkeit $0.05$  den Fehler $A$, mit Wahrscheinlichkeit $0.01$ beide Fehler und mit  Wahrscheinlichkeit $0.03$ nur den Fehler $B$ hat.
\begin{enumerate}
\item Mit welcher Wahrscheinlichkeit hat ein Werkstück den Fehler $B$?
\item Mit welcher Wahrscheinlichkeit ist das Werkstück fehlerfrei, beziehungsweise fehlerhaft?
\item Bei einem Werkstück wurde der Fehler $A$ festgestellt, während die Untersuchung auf Fehler $B$ noch nicht erfolgt ist. Mit welcher Wahrscheinlichkeit hat es auch den Fehler $B$?
\end{enumerate}
\end{block}
 \end{frame}

\begin{frame}
    \frametitle{Übungsaufgaben}
\framesubtitle{}
\begin{block}{Aufgabe}
\begin{enumerate}
\item Mit welcher Wahrscheinlichkeit ist ein Werkstück fehlerfrei, falls es den Fehler $B$ nicht besitzt?
\item Sind die Ereignisse "Werkstück hat Fehler $A$" und "Werkstück hat Fehler $B$" unabhängig?
\end{enumerate}
\end{block}
 \end{frame}


\begin{frame}
    \frametitle{Übungsaufgaben}
\framesubtitle{}
\begin{block}{Aufgabe}

In einer Urne befinden sich 4 schwarze und 6 weiße Kugeln.
Es werden nacheinander zwei Kugeln gezogen, wobei die erste Kugel zurückgelegt wird, bevor die Zweite gezogen wird.
Zeigen Sie, dass das Ziehen einer schwarzen oder weißen Kugel im zweiten Zug stochastisch unabhängig davon ist, welche Kugel im ersten Zug gezogen wurde.
Gilt das auch, wenn nach dem ersten Zug die Kugel nicht zurückgelegt wird?


\end{block}
 \end{frame}




\end{document}
