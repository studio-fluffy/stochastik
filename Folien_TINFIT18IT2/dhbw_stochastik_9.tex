\documentclass{beamer}
\usetheme{Warsaw}

\usepackage[utf8]{inputenc}
\usepackage{fancybox}
\usepackage{multimedia} 
\usepackage{subfig}
\usepackage{amsmath}
\usepackage{hyperref}
\usepackage[all]{xy}
\begin{document}


\title[Stochastik] % (optional, only for long titles)
{Stochastik für Informatiker
\\
\includegraphics[scale=0.5]{img/craps}
}
\subtitle{}
\author[Dr. Johannes Riesterer] % (optional, for multiple authors)
{Dr.  rer. nat. Johannes Riesterer}

\date[KPT 2004] % (optional)
{}

\subject{Stochastik}

\frame{\titlepage}






\begin{frame}
    \frametitle{Statistik - Hypothesentest}
\framesubtitle{}

\begin{block}{Parametertests im Gaußmodell}
Gegeben sei in diesem Abschnitt immer das zweiparametrige Gauß'sche Produktmodell
\begin{align*}
(\mathbb{R}^n, \mathcal{B}(\mathbb{R}^n), \prod N(m,v): m \in \mathbb{R}, v > 0 )
\end{align*}
\end{block}

\begin{block}{Parametertests im Gaußmodell}
Modell beschreibt zum Beispiel  $n$-maliges unabhängiges  Messen einer Messgröße (z.B. Temperatur) mit einem Sensor mit unbekannter Qualität. 
\end{block}


 \end{frame}


\begin{frame}
    \frametitle{Statistik - Hypothesentest}
\framesubtitle{}

\begin{block}{Parametertests im Gaußmodell - Chiquadrat-Test}
Wir möchten für das Testproblem der Varianz $v$ 
\begin{align*}
H_0: v \leq v_0 \text{ gegen } H_1: v > v_0
\end{align*}
mit der Entscheidungsfunktion 
 eine geeignete Statistik $T$ und die Konstanten $c$ finden, so dass 
der Test optimal ist und ein  Signifikanzniveau von $\alpha$ besitzt. 
%$\Theta_0 = \mathbb{R} \times [0, v_0]$ und $\Theta_1 = \mathbb{R} \times (v_0, \infty)$ 
\end{block}
\begin{block}{Parametertests im Gaußmodell - Chiquadrat-Test}
Zum Beispiel möchte man testen, ob ein Sensor eine bestimmte Qualität hat (wenig streut).
\end{block}


 \end{frame}


\begin{frame}
    \frametitle{Statistik - Hypothesentest}
\framesubtitle{}

\begin{block}{Student's Satz}
Gegeben sei das Gauß'sche Produktmodell und die Schätzer $M =  \frac{1}{n} \sum_{i= 1}^n X_i$ und $V* = \frac{1}{n-1} \sum_{i= 1}^n (X_i - M)^2$ für Erwartungswert und Varianz. Dann gilt:
\begin{itemize}
\item $M$ und $V^*$ sind stoch. unabhängig.
\item $M$ hat die Verteilung $N(m ,\frac{v}{n})$.
\item $V^*$ hat die Verteilung $\chi^2_{n-1}$ (\href{https://de.wikipedia.org/wiki/Chi-Quadrat-Verteilung}{\underline{Chiquadrat-Verteilung}}).
\item $T_m := \frac{\sqrt{n}(M-m)}{\sqrt{V^*}}$ hat die Verteilung $t_{n-1}$ (\href{https://de.wikipedia.org/wiki/Studentsche_t-Verteilung}{\underline{Student'sche t-Verteilung}}).
\end{itemize}
\end{block}
\begin{figure}[htp]
      \centering
    \includegraphics[width=0.3\textwidth]{img/Guinness}

      \caption{Quelle: Wikipedia}
\end{figure}


 \end{frame}

\begin{frame}
    \frametitle{Statistik - Hypothesentest}
\framesubtitle{}

\begin{block}{Beispiel-Beweis: Quadrat einer Standardnormalverteilten Zufallsvariable}
Sei $\phi_{0,1}$ die Dichte der Standardnormalverteilung.
Aus Symmetriegründen hat $|X|$ die Dichte $2 \phi_{0,1}$ auf $\mathcal{X} = (0, \infty)$ (den Fall $X=0$ kann man ignorieren, da er mit Wahrscheinlichkeit $0$ auftritt). Durch \href{https://de.wikipedia.org/wiki/Integration_durch_Substitution}{\underline{Substitution}}    $\varphi(x)= x^2$ mit Umkehrfunktion $\varphi(y)^{-1} = \sqrt{y}$ hat $X^2 = \varphi(|X|)$ die Dichte
\begin{align*}
\rho_{X^2} (y) = 2 \phi_{0,1}(\sqrt{y}) \frac{1}{2}y^{-\frac{1}{2}} = \frac{1}{\sqrt{2 \pi}} e^{-\frac{y}{2}} y^{-\frac{1}{2}} = \frac{\Gamma(\frac{1}{2})}{\sqrt{\pi}} \gamma_{\frac{1}{2}, \frac{1}{2}} (y)
\end{align*}
\end{block}
 \end{frame}


\begin{frame}
    \frametitle{Statistik - Hypothesentest}
\framesubtitle{}

\begin{block}{Chiquadrat-Test}
Für den Schätzer $V^*= \sum_{i=1}^{n} (X_i - M)^2$ der Varianz 
ist der Test mit dem Ablehnungsbereich $\{ \sum_{i=1}^{n} (X_i - M)^2 > v_0 \chi^2_{n-1;1-\alpha} \}$, wobei $ \chi^2_{n-1;1-\alpha}$ ein $\alpha$-Fraktil der $\chi^2_{n-1}$-Verteilung ist,   ein bester Test zum Signifikanzniveau $\alpha$.
\end{block}
 \end{frame}


\begin{frame}
    \frametitle{Statistik - Hypothesentest}
\framesubtitle{}

\begin{block}{Chiquadrat-Test}
Für den Schätzer $M =  \frac{1}{n} \sum_{i= 1}^n X_i$ für den Erwartungswert 
ist der Test mit dem Ablehnungsbereich 
$\{ \frac{M -m_0}{\sqrt{\frac{n}{V^*}}}  >  t_{n-1; 1- \alpha} \}$,
 wobei $t_{n-1; 1- \alpha} $ ein $\alpha$-Frakitl der $t_{n-1}$-Verteilung ist,   ein bester Test zum Signifikanzniveau $\alpha$.
\end{block}
 \end{frame}



\end{document}
